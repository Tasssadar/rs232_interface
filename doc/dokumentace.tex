\documentclass[12pt, a4paper, oneside]{article} 
% velikost písma, stránky, typ dokumentu -- detaily viz literatura

\usepackage{ucs}
\usepackage{czech} % nastavení češtiny
%\usepackage[latin2]{inputenc}
%\usepackage[cp1250]{inputenc} % pro win1250
\usepackage[center]{caption} 
\usepackage[utf8]{inputenc}
\usepackage{wrapfig} % nastavení obtékání textu
\usepackage{graphicx,amsmath} % nastavení grafiky, matematiky
\usepackage{subfig} % více obrázků vedle sebe 
\usepackage{float}
\usepackage{amsmath}
\usepackage{amssymb}
\usepackage{bbding}
\usepackage{enumitem}
\usepackage{breakurl}
\usepackage{pdflscape}
\usepackage{wrapfig}

%\usepackage{indentfirst}

\usepackage{tocloft} %přidá tečky do obsahu ke kapitolám /sekcím 
\renewcommand{\cftsecdotsep}{\cftdotsep}

\usepackage[bookmarksopen,colorlinks,plainpages=false,linkcolor=black,urlcolor=blue,citecolor=black,filecolor=black,menucolor=black,unicode=true]{hyperref}

\urlstyle{rm}
%bookmarksopen -- open up bookmark tree 
%colorlinks -- zbarví odkazy (implicitně orámovaný nezbarvený text)
%urlcolor -- barva odkazů (implicitně magenta) 
%linkcolor=black -- barva odkazů v obsahu (implicitně red)

\usepackage{listings}
\usepackage{color}
\definecolor{lightgray}{RGB}{240,240,240}
\definecolor{darkgray}{rgb}{.4,.4,.4}
\definecolor{purple}{rgb}{0.65, 0.12, 0.82}
\definecolor{darkgreen}{RGB}{0,150,0}

\lstdefinelanguage{JavaScript}{
  keywords={typeof, new, true, false, catch, function, return, null, catch, switch, var, if, in, while, do, else, case, break, for},
  keywordstyle=\color{blue}\bfseries,
  ndkeywords={class, export, boolean, throw, implements, import, this},
  ndkeywordstyle=\color{blue}\bfseries,
  identifierstyle=\color{black},
  sensitive=zr,
  comment=[l]{//},
  morecomment=[s]{/*}{*/},
  commentstyle=\color{darkgreen}\ttfamily,
  stringstyle=\color{red}\ttfamily,
  morestring=[b]',
  morestring=[b]"
}

\lstset{
   backgroundcolor=\color{lightgray},
   extendedchars=true,
   basicstyle=\footnotesize\ttfamily,
   showstringspaces=false,
   showspaces=false,
   numbers=left,
   numberstyle=\footnotesize,
   numbersep=9pt,
   tabsize=2,
   breaklines=true,
   showtabs=false,
   aboveskip=5mm,
   belowskip=7mm,
   captionpos=b
}

\renewcommand{\listingscaption}{Příklad}
\renewcommand{\listoflistingscaption}{Příklady}

% \usepackage{parskip} -- zapne americké odstavce v celé práci

%\addtolength{\textwidth}{-2mm} 
%\addtolength{\hoffset}{4mm}  % posun textu kvůli kroužkové vazbě  

\setlength{\intextsep}{5mm} % nastavení mezery okolo obrázků

% nastavení příkazu >\figcaption pro popis čehokoli, jako by to byly obrázky 
\makeatletter   
\newcommand\figcaption{\def\@captype{figure}\caption}
\makeatother

\renewcommand\refname{Literatura} 
%\def\bibname{PŘÍLOHA D: Reference}
%\renewcommand\bibname{PŘÍLOHA D: Reference}
% přejmenuje anglický název Reference na české Literatura


%\makeindex % příprava pro výrobu indexu (jestli ho chcete)

%%    VLNKA <fileinput>  KkSsVvZzOoUuAaIi        
% Defaultni  koncovka pro <fileinput> je  ".tex"
%FIXME: haze error
%\cstieon % Vypne chovani vlnky jako tvrde mezery v matematickem rezimu

%%%%%%%%%%%%%%%%%%%%%%%%%%%%%%%%%%%%%%%%%%%%%%%%%%%%%%%%%%%%%%%
%V PROSTŘEDÍ ROVNIC SE NESMÍ VYSKYTOVAT PRÁZDNÝ ŘÁDEK
%
%PROGRAMY VLNKA A CSINDEX SE MUSÍ SPUSTIT SAMOSTATNĚ
%%%%%%%%%%%%%%%%%%%%%%%%%%%%%%%%%%%%%%%%%%%%%%%%%%%%%%%%%%%%%%%

% definice příkazů 
\newcommand{\D}{\medskip \noindent} % nový odstavec v "americkém" formátování 
\newcommand{\B}{\textbf} %tučné písmo
\newcommand{\A}{\mathbf} %tučné písmo v matematickém režimu
\newcommand{\TO}{\ensuremath{\boldsymbol\Omega}} % tučný znak velké omega -- pro ohmy
\newcommand{\I}{\index}  % vytváří položku indexu (asi nepoužijete)
\newcommand{\Deg}[1][]{\ensuremath{{#1}^\circ}} % vysází značku stupně Celsia
\newcommand{\Def}{\footnotesize Definice: \normalsize}
\newcommand{\Pos}{\footnotesize Experiment: \normalsize}
\newcommand{\Odv}{\footnotesize Odvození: \normalsize}
\newcommand{\Vym}{\footnotesize Vymezení pojmu: \normalsize}
\newcommand{\Ob}{obrázek }
\newcommand{\It}{\textit}  % kurzíva
\newcommand{\M}{\mathrm}   % v prostředí rovnic nastaví normální písmo (místo kurzívy ) 
\newcommand{\F}{\footnotesize} % zmenšená velikost písma
\newcommand{\N}{\normalsize} % normální velikost písma
%\newcommand{\U}{\underline}  % podtržené písmo
\newcommand{\e}{\ensuremath} 
\newcommand{\Has}{\textcolor{green}{\CheckmarkBold}}
\newcommand{\NoHas}{\textcolor{red}{\XSolidBrush}}
% další příkaz se aplikuje, pouze, když jste v matematickém režimu

%\hyphenation{Pusť-me pla-tí hod-no-ty do-sa-dí-me za-da-né dal-ším}
% dělení slov, kdyby implicitní nevyhovovalo

\linespread{1.2} 
% řádkování 1,5x  
% použijete podle situace  

\addtolength{\textheight}{30pt}
\unitlength=1mm % nastavení volby jednotek 

% konec hlavičky
%%%%%%%%%%%%%%%%%%%%%%%%%%%%%%%%%%%%%%%%%%%%%%%%%%%%%%%%%%%%%%%%%%%
%%%%%%%%%%%%%%%%%%%%%%%%%%%%%%%%%%%%%%%%%%%%%%%%%%%%%%%%%%%%%%%%%%%

\pagestyle{empty}

\begin{document}
\setlength{\voffset}{-20mm}
\begin{center}

{\Large\B{Komunikace pomocí sériové linky}} \\
{\large Vojtěch Boček, učo 433572, 433572@mail.muni.cz} \\
{Projekt do předmětu PB170 -- Seminář z konstrukce digitálních systémů} \\

\end{center}

\section{Popis projektu}
Projekt spočívá v navrhnutí obvodu pro vývojovou desku s FPGA, který bude zvládat následující:
\begin{itemize}
    \item přímat a posílat data pomocí sériové linky
    \item parsovat a sestavovat jednoduchý binární protokol
    \item reagovat na příkazy z počítače (např. \uv{rozsviť LED č. 1})
    \item posílat informace o změnách vstupů na desce (např. při stisknutí tlačítka)
\end{itemize}

\section{Implementace}
Většina projektu (přijmač i vysílač sériové linky, hlavní kontrolér) je implementována pomocí jednoduchého stavového automatu.
\subsection{Soubory projektu}
\begin{itemize}
    \item \verb-clkdiv.v- -- jednoduchá dělička hodinového signálu, používaná pro přijmač a vysílač sériové linky.
    \item \verb-hex_num.v- -- modul pro ovládání jedné HEX číslice na vývojové desce. Jako vstup slouží číslo od 0 do 10, kde 0 až 9 zobrazí korespondující číslice a 10 slouží jako zástupná hodnota pro mínus.
    \item \verb-uart_rx.v- -- příjmač sériové linky. Vzorkuje příchozí pin na 16 násobku rychlosti linky a pokud detekuje start bit (tj. hodnotu 0), začne přijmat byte do registru \verb-RX_DATA-. Přijmutí bytu signalizuje přepnutím registuru \verb-RX_RECV- na hodnotu 1, a registr \verb-RX_DATA- je po tuto dobu možné vyčíst.
    \item \verb-uart_tx.v- -- vysílač sériové linky. Po nastavení vstupního registru \verb-TX_START- na 1 začne odesílat byte z registru \verb-TX_DATA- po sériové lince. Po dobu přenosu je registr \verb-TX_BUSY- nastaven na hodnotu 1.
    \item \verb-rs232_interface_top.v- -- hlavní kontrolér. Při každém tiku hodinového signálu se kontrolují příchozí byty ze sériové linky, stav tlačítek a spínačů a vykoná se adekvátní akce, nebo se odesílají byty směrem do počítače.
\end{itemize}

\subsection{Protokol}
Protokol používá řazení bytů little-endian. Packet má vždy 3 bytovou hlavičku se start bytem (vždy \verb-0xFF-), číslem příkazu a délkou dat, která následují za hlavičkou. Za hlavičkou následují data, pokud jsou nějaká odeslaná. Příklad jednoho packetu:
\begin{verbatim}
  0            1        2       3-...
| start byte | příkaz | délka | data
  0xFF         0x00     0x01    0x02
\end{verbatim}
\begin{tabular}{ | l | l | l | p{8cm} | }
    \hline
    \B{jméno}      & \B{č. příkazu} & \B{délka} & \B{popis} \\ \hline

    \multicolumn{4}{|c|}{Příkazy směrem do počítače.} \\ \hline
    \verb-CMD_KEY- & \verb-0x00-    & 1         & Odeslán při změně stavu tlačítek. Obsahuje 1~byte, kde první 4 bity signalizují, zda je 1.-4. tlačítko stisknuto. \\ \hline
    \verb-CMD_SW-  & \verb-0x01-    & 3         & Odeslán při změně stavu přepínačů. Obsahuje 3~byty, ve kterých po složení prvních 18~bitů signalizuje, zda je 1.-18. přepínač v poloze 1. \\ \hline

    \multicolumn{4}{|c|}{Příkazy směrem do desky.} \\ \hline
    \verb-CMD_LEDG- & \verb-0x02-    & 1         & Zapne/vypne červené LED na desce. Každý bit ve vstupním bytu je jedna LED. \\ \hline
    \verb-CMD_LEDR- & \verb-0x03-    & 3         & Zapne/vypne zelené LED na desce. Prvních 18 bitů ve vstupních 3 bytech jsou jednotlivé LED. \\ \hline
    \verb-CMD_SET_HEX- & \verb-0x04- & 2         & Nastaví hodnotu HEX číslice. První byte je index číslice (0 až 7), druhý je hodnota k zobrazení (0 až 10, 10 znamená mínus). \\ \hline
\end{tabular}

\end{document}
